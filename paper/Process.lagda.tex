\begin{code}[hide]
open import Data.Bool using (Bool; true; false; if_then_else_)
open import Data.Product using (_×_; _,_; ∃; ∃-syntax)
open import Relation.Binary.PropositionalEquality using (_≡_; refl; cong; cong₂)
open import Data.List.Base using (List; []; _∷_; [_]; _++_)

open import Type
open import Context
open import Permutations
\end{code}

\subsection{Process Representation}
\label{sec:process-agda}

CO-DE BRUIJN INDICES

We adopt an \emph{intrinsically typed} representation of processes, so that a
process is also a proof derivation showing that the process is well typed. This
choice increases the effort in the definition of the datatypes for processes and
their operational semantics, but it pays off in the rest of the formalization
for a number of reasons:

\begin{itemize}
\item channel names are encoded in the context splitting proofs occurring within
  processes, hence they do not need to be represented explicitly.
\item processes and typing rules are conflated in the same datatype, thus
  reducing the overall number of datatypes in the formalization;
\item the typing preservation results are embedded in the definition of the
  operational semantics of processes and do not need to be proved separately.
\end{itemize}

We now illustrate the definition of the \AgdaDatatype{Process} datatype and
describe each constructor in detail. The datatype is indexed by a typing
context:

\begin{AgdaAlign}
\begin{code}
data Process : Context → Set where
   link      : ∀{Γ A B} (d : Dual A B) (p : Γ ≃ [ A ] + [ B ]) → Process Γ
\end{code}

A \AgdaInductiveConstructor{link} process is well typed in a typing context with
exactly two types $A$ and $B$ which must be related by duality. We use a proof
$p$ that $\Context \simeq [A] + [B]$ instead of requiring $\Context$ to be
simply $[A,B]$ so that it becomes straightforward to define \SLink using
\AgdaFunction{dual-symm} and \AgdaFunction{+-comm}.

\begin{code}
   fail      : ∀{Γ Δ} (p : Γ ≃ ⊤ , Δ) → Process Γ
\end{code}

A \AgdaInductiveConstructor{fail} process simply requires the presence of $\Top$
in the typing context.

\begin{code}
   close     : Process [ 𝟙 ]
\end{code}

A \AgdaInductiveConstructor{close} process is well typed in the singleton
context $[\One]$.

\begin{code}
   wait      : ∀{Γ Δ} (p : Γ ≃ ⊥ , Δ) → Process Δ → Process Γ
\end{code}

A \AgdaInductiveConstructor{wait} process is well typed in any context
$\ContextC$ that contains $\Bot$ and whose residual $\ContextD$ allows the
continuation to be well typed.

\begin{code}
   select    : ∀{Γ Δ A B} (x : Bool) (p : Γ ≃ A ⊕ B , Δ) →
               Process ((if x then A else B) ∷ Δ) → Process Γ
\end{code}

We represent the tags $\InTag_1$ and $\InTag_2$ with Agda′s boolean values
\AgdaInductiveConstructor{true} and \AgdaInductiveConstructor{false}. In this
way we can use a single constructor \AgdaInductiveConstructor{select} for the
selections $\Select\x{\InTag_i}.P$. According to the semantics of \Calculus, the
channel on which the selection is performed is consumed and the continuation
process uses a fresh continuation channel. As a consequence, the type of the
continuation channel, which is either $A$ or $B$ depending on the value of the
tag $x$, is \emph{prepended} in front of the typing context used to type the
continuation process.

\begin{code}
   case      : ∀{Γ Δ A B} (p : Γ ≃ A & B , Δ) →
               Process (A ∷ Δ) → Process (B ∷ Δ) → Process Γ
\end{code}

A \AgdaInductiveConstructor{case} process has two possible continuations. Note
again that the continuation channel, which is received along with the tag, has
either type $A$ or type $B$ depending on the tag and its type is prepended to
the typing context of the continuation process since it is the newest channel.

\begin{code}
   fork      : ∀{Γ Δ Γ₁ Γ₂ A B} (p : Γ ≃ A ⊗ B , Δ) (q : Δ ≃ Γ₁ + Γ₂) →
               Process (A ∷ Γ₁) → Process (B ∷ Γ₂) → Process Γ
\end{code}

A \AgdaInductiveConstructor{fork} process creates two fresh continuation
channels and spaws two continuation processes. Since context splitting is a
ternary relation, we need two splitting proofs $p$ and $q$ to isolate the type
$A \Ten B$ of channel on which the pair of continuation channels is sent and to
distribute the remaining channel across the two continuation processes.

\begin{code}
   join      : ∀{Γ Δ A B} (p : Γ ≃ A ⅋ B , Δ) →
               Process (B ∷ A ∷ Δ) → Process Γ
\end{code}

The \AgdaInductiveConstructor{join} process has a single continuation which uses
the pair of continuation channels received from a channel of type $A \Par B$.
Note that $B$ is prepended in front of $A$, somewhat implying that the second
continuation is bound \emph{after} the first even though, from a technical
standpoint, both continuations are bound simultaneously. We could have prepended
the types $A$ and $B$ also in the opposite order, provided that the reduction
rule \RFork (formalized in \Cref{sec:semantics-agda}) is suitably adjusted.

\begin{code}
   server    : ∀{Γ Δ A} (p : Γ ≃ ¡ A , Δ) (un : Un Δ) →
               Process (A ∷ Δ) → Process Γ
\end{code}

A \AgdaInductiveConstructor{server} constructor introduces a channel of type
$\OfCourse A$ given a continuation process that consumes a channel of type $A$
and a proof $\mathit{un}$ that every other channel in $\ContextD$ is
unrestricted.\Luca{Introdurre terminologia e dire che unrestricted vuol dire che
il canale può essere ``duplicato'' e scartato.}

\begin{code}
   client    : ∀{Γ Δ A} (p : Γ ≃ ¿ A , Δ) → Process (A ∷ Δ) → Process Γ
   weaken    : ∀{Γ Δ A} (p : Γ ≃ ¿ A , Δ) → Process Δ → Process Γ
   contract  : ∀{Γ Δ A} (p : Γ ≃ ¿ A , Δ) → Process (¿ A ∷ ¿ A ∷ Δ) → Process Γ
\end{code}

The constructors \AgdaInductiveConstructor{client},
\AgdaInductiveConstructor{weaken} and \AgdaInductiveConstructor{contract}
introduce a channel of type $\WhyNot A$ and follow the same pattern of the
previous process forms. Note that, in the case of
\AgdaInductiveConstructor{contract}, the contraction of $\WhyNot A$ applies to
the first two channels in the typing context of the continuation. This is
consistent with the fact that \Calculus uses an explicit prefix
$\Contract\x\y\z$ that models contraction and binds $y$ and $z$ in the
continuation process. In a pen-and-paper presentation, like the one by
Wadler~\citep{Wadler14}, it is arguably more convenient to have weaking and
contraction typing rules without the corresponding process forms.

\begin{code}
   cut       : ∀{Γ Γ₁ Γ₂ A B} (d : Dual A B) (p : Γ ≃ Γ₁ + Γ₂) →
               Process (A ∷ Γ₁) → Process (B ∷ Γ₂) → Process Γ
\end{code}

A \AgdaInductiveConstructor{cut} process incorporates a duality proof for the
peer endpoints of the channel being restricted and a splitting proof that
distributes the free channels among the two parallel processes.

\end{AgdaAlign}

\begin{code}
#process : ∀{Γ Δ} → Γ # Δ → Process Γ → Process Δ
\end{code}
\begin{code}[hide]
#process π (link d p) with #one+ π p
... | Δ′ , q , π′ with #singleton-inv π′
... | refl = link d q
#process π close with #singleton-inv π
... | refl = close
#process π (fail p) with #one+ π p
... | Δ′ , q , π′ = fail q
#process π (wait p P) with #one+ π p
... | Δ′ , q , π′ = wait q (#process π′ P)
#process π (select x p P) with #one+ π p
... | Δ′ , q , π′ = select x q (#process (#next π′) P)
#process π (case p P Q) with #one+ π p
... | Δ′ , q , π′ = case q (#process (#next π′) P) (#process (#next π′) Q)
#process π (fork p q P Q) with #one+ π p
... | Δ′ , p′ , π′ with #split π′ q
... | Δ₁ , Δ₂ , q′ , π₁ , π₂ = fork p′ q′ (#process (#next π₁) P) (#process (#next π₂) Q)
#process π (join p P) with #one+ π p
... | Δ′ , q , π′ = join q (#process (#next (#next π′)) P)
#process π (cut d p P Q) with #split π p
... | Δ₁ , Δ₂ , q , π₁ , π₂ = cut d q (#process (#next π₁) P) (#process (#next π₂) Q)
#process π (server p un P) with #one+ π p
... | Δ′ , q , π′ = server q (#un π′ un) (#process (#next π′) P)
#process π (client p P) with #one+ π p
... | Δ′ , q , π′ = client q (#process (#next π′) P)
#process π (weaken p P) with #one+ π p
... | Δ′ , q , π′ = weaken q (#process π′ P)
#process π (contract p P) with #one+ π p
... | Δ′ , q , π′ = contract q (#process (#next (#next π′)) P)
\end{code}

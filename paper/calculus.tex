\section{The Calculus}
\label{sec:calculus}

\begin{table}
    \caption{Syntax of the calculus of pure sessions.}
    \label{tab:syntax}
    \begin{math}
        \displaystyle
        \begin{array}[t]{rcll}
            P, Q & ::= & \Link\x\y \\
                & | & \Close\x \\
                & | & \Fail\x \\
                & | & \Wait\x.P \\
                & | & \Select\x{\InTag_i}\z.P \\
                & | & \Case\x\z{P}{Q} \\
                & | & \Fork\x\y\z{P}{Q} \\
                & | & \Join\x\y\z.P \\
                & | & \Cut\x{P}{Q} \\
        \end{array}
    \end{math}
\end{table}

The calculus of pure sessions that is formalized in our development is closely
related to the linear fragment of GV~\citep{Wadler14}, without shared channels.
Its syntax makes use of an infinite set of \emph{channel names}, ranged over by
$x$, $y$ and $z$, and is shown in \cref{tab:syntax}.

\[
    \begin{array}{rrcll}
      \SLink &
      \Link\x\y & \pcong & \Link\y\x \\
      \SComm &
      \Cut\x{P}{Q} & \pcong & \Cut\x{Q}{P} \\
      \SWait &
      \Cut\x{\Wait\y.P}{Q} & \pcong & \Wait\y.\Cut\x{P}{Q} & x \ne y \\
      \SFail &
      \Cut\x{\Fail\y}{P} & \pcong & \Fail\y & x \ne y \\
      \SSelect &
      \Cut\x{\Select\y{\InTag_i}\z.P}{Q} & \pcong & \Select\y{\InTag_i}\z.\Cut\x{P}{Q} & x \ne y,z \\
      \SCase &
      \Cut\x{\Case\y\z{P}{Q}}{R} & \pcong & \Case\y\z{\Cut\x{P}{R}}{\Cut\x{Q}{R}} & x \ne y,z \\
      \SForkL &
      \Cut\x{\Fork\y\u\v{P}{Q}}{R} & \pcong & \Fork\y\u\v{\Cut\x{P}{R}}{Q} & x \in \fn{P}\setminus\set{y,u,v} \\
      \SForkR &
      \Cut\x{\Fork\y\u\v{P}{Q}}{R} & \pcong & \Fork\y\u\v{P}{\Cut\x{Q}{R}} & x \in \fn{Q}\setminus\set{y,u,v} \\
      \SJoin &
      \Cut\x{\Join\y\u\v.P}{Q} & \pcong & \Join\y\u\v.\Cut\x{P}{Q} & x \ne y,u,v \\
      \\
      \RLink &
      \Cut\x{\Link\x\y}{P} & \red & P\subst\y\x \\
      \RFail &
      \Cut\x{\Fail\y}{P} & \red & \Fail\y & x \ne y \\
      \RClose &
      \Cut\x{\Close\x}{\Wait\x.P} & \red & P \\
      \RSelect &
      \Cut\x{\Select\x{\Tag_i}\y.P}{\Case\x\y{Q_1}{Q_2}} & \red & \Cut\y{P}{Q_i} & i\in\set{1,2} \\
      \RFork &
      \Cut\x{\Fork\x\y\z{P}{Q}}{\Join\x\y\z.R} & \red & \Cut\y{P}{\Cut\z{Q}{R}} \\
      \RCut &
      \Cut\x{P}{R} & \red & \Cut\x{Q}{R} & P \red Q \\
      \RCong &
      P & \red & Q & P \pcong R \red Q \\
    \end{array}
\]


\section{Agda Formalization}
\label{sec:formalization}

\begin{code}[hide]
open import Data.Product using (_×_; _,_; ∃; ∃-syntax)
open import Relation.Binary.PropositionalEquality using (_≡_; refl; cong; cong₂)
open import Data.List.Base using (List; []; _∷_; [_]; _++_)
\end{code}

\subsection{Type Representation}
\label{sec:type-agda}

The representation of types is standard, with one constructor for each of the
forms described in \Cref{sec:calculus}.

\begin{code}
data Type : Set where
  𝟘 𝟙 ⊥ ⊤          : Type
  ¡ ¿              : Type → Type
  _&_ _⊕_ _⊗_ _⅋_  : Type → Type → Type
\end{code}

The notion of duality is formalized as a \emph{relation} \AgdaDatatype{Dual}
such that $\AgdaDatatype{Dual}~A~B$ holds if and only if $A = \dual{B}$.

\begin{code}
data Dual : Type → Type → Set where
  d-𝟘-⊤  : Dual 𝟘 ⊤
  d-⊤-𝟘  : Dual ⊤ 𝟘
  d-𝟙-⊥  : Dual 𝟙 ⊥
  d-⊥-𝟙  : Dual ⊥ 𝟙
  d-!-?  : ∀{A B} → Dual A B → Dual (¡ A) (¿ B)
  d-?-!  : ∀{A B} → Dual A B → Dual (¿ A) (¡ B)
  d-&-⊕  : ∀{A B A′ B′} → Dual A A′ → Dual B B′ → Dual (A & B) (A′ ⊕ B′)
  d-⊕-&  : ∀{A B A′ B′} → Dual A A′ → Dual B B′ → Dual (A ⊕ B) (A′ & B′)
  d-⊗-⅋  : ∀{A B A′ B′} → Dual A A′ → Dual B B′ → Dual (A ⊗ B) (A′ ⅋ B′)
  d-⅋-⊗  : ∀{A B A′ B′} → Dual A A′ → Dual B B′ → Dual (A ⅋ B) (A′ ⊗ B′)
\end{code}

It is straightforward to prove that duality is a symmetric relation and that it
behaves as an involution. From this we prove that it acts as the function
$\dual\cdot$ defined in \Cref{sec:calculus}.

\begin{code}
dual-symm   : ∀{A B} → Dual A B → Dual B A
dual-inv    : ∀{A B C} → Dual A B → Dual B C → A ≡ C
dual-fun-r  : ∀{A B C} → Dual A B → Dual A C → B ≡ C
dual-fun-l  : ∀{A B C} → Dual B A → Dual C A → B ≡ C
\end{code}
\begin{code}[hide]
dual-symm d-𝟘-⊤ = d-⊤-𝟘
dual-symm d-⊤-𝟘 = d-𝟘-⊤
dual-symm d-𝟙-⊥ = d-⊥-𝟙
dual-symm d-⊥-𝟙 = d-𝟙-⊥
dual-symm (d-!-? p) = d-?-! (dual-symm p)
dual-symm (d-?-! p) = d-!-? (dual-symm p)
dual-symm (d-&-⊕ p q) = d-⊕-& (dual-symm p) (dual-symm q)
dual-symm (d-⊕-& p q) = d-&-⊕ (dual-symm p) (dual-symm q)
dual-symm (d-⊗-⅋ p q) = d-⅋-⊗ (dual-symm p) (dual-symm q)
dual-symm (d-⅋-⊗ p q) = d-⊗-⅋ (dual-symm p) (dual-symm q)
\end{code}

\begin{code}[hide]
dual-inv d-𝟘-⊤ d-⊤-𝟘 = refl
dual-inv d-⊤-𝟘 d-𝟘-⊤ = refl
dual-inv d-𝟙-⊥ d-⊥-𝟙 = refl
dual-inv d-⊥-𝟙 d-𝟙-⊥ = refl
dual-inv (d-!-? p) (d-?-! q) = cong ¡ (dual-inv p q)
dual-inv (d-?-! p) (d-!-? q) = cong ¿ (dual-inv p q)
dual-inv (d-&-⊕ p q) (d-⊕-& r s) = cong₂ _&_ (dual-inv p r) (dual-inv q s)
dual-inv (d-⊕-& p q) (d-&-⊕ r s) = cong₂ _⊕_ (dual-inv p r) (dual-inv q s)
dual-inv (d-⊗-⅋ p q) (d-⅋-⊗ r s) = cong₂ _⊗_ (dual-inv p r) (dual-inv q s)
dual-inv (d-⅋-⊗ p q) (d-⊗-⅋ r s) = cong₂ _⅋_ (dual-inv p r) (dual-inv q s)

dual-fun-r d e = dual-inv (dual-symm d) e

dual-fun-l d e = dual-inv d (dual-symm e)
\end{code}

\subsection{Context Representation}
\label{sec:context-agda}

We are going to adopt a nameless representation of channels whereby a channel is
identified by its position in a typing context. This representation is akin to
using De Bruijn indices, except that the index, instead of being represented
explicitly by a natural number, is computable from the \emph{proof} that the
type of the channel belong to the typing context.
%
Typing contexts are represented using \emph{lists} of types, where the
(polymorphic) \AgdaDatatype{List} type is defined in Agda′s standard library.

\begin{code}
Context : Set
Context = List Type
\end{code}

The most important operation concerning typing contexts is \emph{splitting}. The
splitting of $\ContextC$ into $\ContextD$ and $\ContextE$, which is denoted by
$\Splitting\ContextC\ContextD\ContextE$, represents the fact that $\ContextC$
contains all the types contained in $\ContextD$ and $\ContextE$, preserving both
their overall \emph{multiplicity} and also their relative \emph{order} within
$\ContextD$ and $\ContextE$. A \emph{proof} of
$\Splitting\ContextC\ContextD\ContextE$ shows how the types in $\ContextC$ are
distributed in $\ContextD$ and $\ContextE$ from left to right.

\begin{code}[hide]
infix 4 _≃_+_ _≃_,_
\end{code}
\begin{code}
data _≃_+_ : Context → Context → Context → Set where
  split-e  : [] ≃ [] + []
  split-l  : ∀{A Γ Δ Θ} → Γ ≃ Δ + Θ → A ∷ Γ ≃ A ∷ Δ + Θ
  split-r  : ∀{A Γ Δ Θ} → Γ ≃ Δ + Θ → A ∷ Γ ≃ Δ + A ∷ Θ
\end{code}

As an example, below is the splitting of the context $[A, B, C, D]$ into $[A,
D]$ and $[B, C]$. Note how the sequence of \AgdaInductiveConstructor{split-l}
and \AgdaInductiveConstructor{split-r} applications determines where each of the
types in $[A, B, C, D]$ ends up in $[A, D]$ and $[B, C]$.

\begin{code}[hide]
module _ where
  postulate A B C D : Type
\end{code}
\begin{code}
  splitting-example : (A ∷ B ∷ C ∷ D ∷ []) ≃ (A ∷ D ∷ []) + (B ∷ C ∷ [])
  splitting-example = split-l (split-r (split-r (split-l split-e)))
\end{code}

It is often the case that the context $\ContextD$ in a splitting
$\Splitting\ContextC\ContextD\ContextE$ is a singleton list $[A]$. For this
particular case, we introduce a dedicated notation that allows us to represent
this case in a more compact and readable way, as
$\SimpleSplitting\ContextC{A}\ContextE$.

\begin{code}
_≃_,_ : Context → Type → Context → Set
Γ ≃ A , Δ = Γ ≃ [ A ] + Δ
\end{code}

Context splitting enjoys a number of expected properties. In particular, it is
easy to see that splitting is commutative and that the empty context is both a
left and right unit of splitting.

\begin{code}
+-comm    : ∀{Γ Δ Θ} → Γ ≃ Δ + Θ → Γ ≃ Θ + Δ
+-unit-l  : ∀{Γ} → Γ ≃ [] + Γ
+-unit-r  : ∀{Γ} → Γ ≃ Γ + []
\end{code}
\begin{code}[hide]
+-comm split-e = split-e
+-comm (split-l p) = split-r (+-comm p)
+-comm (split-r p) = split-l (+-comm p)

+-unit-l {[]} = split-e
+-unit-l {_ ∷ _} = split-r +-unit-l

+-unit-r = +-comm +-unit-l
\end{code}

Context splitting is also associative in a sense that is made precise below. If
we write $\ContextD + \ContextE$ for some $\ContextC$ such that $Γ ≃ Δ + Θ$, then
we can prove that $Γ_1 + (Γ_2 + Γ_3) = (Γ_1 + Γ_2) + Γ_3$.

\begin{code}
+-assoc-r  : ∀{Γ Δ Θ Δ′ Θ′} → Γ ≃ Δ + Θ → Θ ≃ Δ′ + Θ′ →
             ∃[ Γ′ ] Γ′ ≃ Δ + Δ′ × Γ ≃ Γ′ + Θ′
+-assoc-l  : ∀{Γ Δ Θ Δ′ Θ′} → Γ ≃ Δ + Θ → Δ ≃ Δ′ + Θ′ →
             ∃[ Γ′ ] Γ′ ≃ Θ′ + Θ × Γ ≃ Δ′ + Γ′
\end{code}
\begin{code}[hide]
+-assoc-r split-e split-e = [] , split-e , split-e
+-assoc-r (split-l p) q with +-assoc-r p q
... | _ , p′ , q′ = _ , split-l p′ , split-l q′
+-assoc-r (split-r p) (split-l q) with +-assoc-r p q
... | _ , p′ , q′ = _ , split-r p′ , split-l q′
+-assoc-r (split-r p) (split-r q) with +-assoc-r p q
... | _ , p′ , q′ = _ , p′ , split-r q′

+-assoc-l p q with +-assoc-r (+-comm p) (+-comm q)
... | Δ , r , p′ = Δ , +-comm r , +-comm p′

+-empty-l : ∀{Γ Δ} → Γ ≃ [] + Δ → Γ ≡ Δ
+-empty-l split-e = refl
+-empty-l (split-r p) = cong (_ ∷_) (+-empty-l p)

+-sing-l : ∀{A B Γ} → [ A ] ≃ B , Γ → A ≡ B × Γ ≡ []
+-sing-l (split-l split-e) = refl , refl
\end{code}

\subsection{Context Permutations}
\label{sec:permutation-agda}

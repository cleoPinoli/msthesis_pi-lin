\documentclass[12pt]{letter}

\usepackage{a4wide}

\begin{document}
\begin{letter}{}
\opening{Dear Editor,}

I am submitting a manuscript entitled ``A Continuation-Based Solution of the
Linearity Challenge'' to the Journal of Automated Reasoning.

This paper addresses the \emph{linearity challenge}, which appears in a recently
published collection of problems concerning the formalisation of concurrent and
distributed models of computation. The ultimate goal of the challenge (and, more
generally, of the benchmark) is to ``foster the adoption of proof mechanisation
in future research on concurrency'' pretty much like the POPLMark challenge did
for sequential models of programming languages. The linearity challenge is
specifically concerned with the formalisation of a calculus of sessions, where
session endpoints are linear resources that cannot be duplicated or discarded.

The paper describes a particular solution of the linearity challenge in which we
formalise a low level model of computation whose type system corresponds to the
proof system of classical linear logic and into which sessions can be encoded
instead of being featured natively. Despite the low level of abstraction of the
calculus and the full range of features supported by its type system, we show
that the calculus allows for a remarkably simple formalisation. The paper
includes a thorough comparison with related formalisations of linear/session
calculi.

The paper has not been published elsewhere. The formalisation started as a
master thesis project of Claudia Raffaelli, a former student of mine and
co-author of the paper. After her graduation, we kept refining and extending the
formalisation once we realised its potential, and eventually decided to write a
paper to share our results and findings with the community.

Thank you for managing our manuscript and considering it for review.

\closing{Best regards,}

Prof. Luca Padovani \\
Department of Computer Science and Engineering \\
University of Bologna \\
ITALY

\end{letter}
\end{document}
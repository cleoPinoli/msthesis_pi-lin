\subsection{Examples}
\label{sec:examples}

\begin{code}[hide]
{-# OPTIONS --rewriting #-}
open import Function using (_$_)
open import Data.Sum using (inj₁; inj₂)
open import Data.Product using (_×_; _,_; ∃; ∃-syntax; curry)
open import Data.Fin using (zero; suc; #_)
open import Data.List.Base using ([]; _∷_; [_])
open import Relation.Unary

open import Type
open import Context
open import Permutations
open import Process
open import DeadlockFreedom using (deadlock-freedom)
\end{code}

In this section we revisit and expand the processes discussed in
\Cref{ex:booleans,ex:echo} and show their encoding in our formalisation. The
encoding of $\Bool$ is straightforward

\begin{code}
𝔹 : Type
𝔹 = 𝟙 ⊕ 𝟙
\end{code}
and the boolean constants are encoded thus:
\begin{code}
True : Proc [ 𝔹 ]
True = select (ch ⟨ < ≫ ⟩ inj₁ (close ch))

False : Proc [ 𝔹 ]
False = select (ch ⟨ < ≫ ⟩ inj₂ (close ch))
\end{code}

We take advantage of the host language for programming higher-order processes.
For example, we can define a conditional process thus:

\begin{code}
If_Else : ∀[ Proc ⇒ Proc ⇒ (dual 𝔹 ∷_) ⊢ Proc ]
If P Else Q = curry∗ case ch (< ≫) (  wait (ch ⟨ < ≫ ⟩ P)
                                   ,  wait (ch ⟨ < ≫ ⟩ Q))
\end{code}

A term $\AgdaFunction{If}~P~\AgdaFunction{Else}~Q$ is a process that waits for a
boolean value (\cf the $\AgdaFunction{dual}~\mathbb{B}$ type at the front of its
typing context) and continues as either $P$ or $Q$ depending on whether it
receives true or false. We use \AgdaFunction{curry∗} (defined in
\Cref{sec:context-agda}) to curry the constructor
\AgdaInductiveConstructor{case} so that we can supply its arguments one by one
saving a few parentheses and reducing clutter (more on this in \Cref{rem:curry}
at the end of this section).

Next we define a process $\AgdaFunction{Drop}~P$ that consumes a boolean and
continues as $P$ regardless of its value.

\begin{code}
Drop : ∀[ Proc ⇒ (dual 𝔹 ∷_) ⊢ Proc ]
Drop P = If P Else P
\end{code}

Using these higher-order forms, it is easy to define the usual boolean
connectives.

\begin{code}
!! : Proc [ 𝔹 ] → Proc [ 𝔹 ]
!! B = curry∗ cut B ≫ (If False Else True)

_&&_ _||_  : Proc [ 𝔹 ] → Proc [ 𝔹 ] → Proc [ 𝔹 ]
A && B   =  curry∗ cut A ≫ $
            curry∗ cut B ≫ $
            If curry∗ link ch (< ≫) ch Else (Drop False)
A || B   =  !! (!! A && !! B)
\end{code}

The function \AgdaFunction{\$} (defined in Agda's standard library) is just a
low-precedence, visible function application operator. We use it as a separator
to flatten deeply nested expressions and save a bunch of parentheses.
%
For the sake of illustration, we have chosen to define the disjunction
\AgdaFunction{||} from the conjunction \AgdaFunction{\&\&} and negation
\AgdaFunction{!!} using De Morgan's laws.

To test our definitions, we implement a simple evaluator using the deadlock
freedom property. We have not proved a termination result, but since linear
logic enjoys cut elimination we can safely annotate the evaluator as
terminating.

\begin{code}
{-# TERMINATING #-}
eval : ∀[ Proc ⇒ Proc ]
eval P with deadlock-freedom P
... | inj₁ (Q , _ , _)  = Q
... | inj₂ (Q , _)      = eval Q
\end{code}

Now if we ask Agda to normalise the goal
$\AgdaFunction{eval}~(\AgdaFunction{False}~\AgdaFunction{||}~\AgdaFunction{False})$
we obtain
$\AgdaInductiveConstructor{select}~(\AgdaInductiveConstructor{ch~⟨~<~•~⟩~inj₂}~(\AgdaInductiveConstructor{close~ch}))$,
that is the definition of \AgdaFunction{False}, as expected.

For the encoding of the polymorphic echo server (\Cref{ex:echo}), we start by
encoding its type $\OfCourse(\forall\X.\dual\X \Par (X \Ten \One))$:

\begin{code}
ServerT : Type
ServerT = `! (`∀ (rav (# 0) ⅋ (var (# 0) ⊗ 𝟙)))
\end{code}

The notation $\AgdaFunction{\#}~n$ (defined in Agda's standard library) creates
an element of \AgdaDatatype{Fin} from the natural number $n$. Here it is used to
create the de Bruijn index of the type variable $X$.
%
We now encode the server
\begin{code}
Server : Proc [ ServerT ]
Server = curry (curry∗ server ch (< ≫)) un-[] $
         curry∗ all ch (< ≫) λ X →
         curry∗ join ch (< ≫) $
         curry∗ (curry∗ fork ch (< ≫)) (curry∗ link ch (< > •) ch) (< ≫) $
         close ch
\end{code}
and the client that sends true to it
\begin{code}
Client : Proc (dual ServerT ∷ 𝔹 ∷ [])
Client = curry∗ client ch (< ≫) $
         curry∗ (ex {_} {𝔹}) ch (< ≫) $
         curry∗ (curry∗ fork ch (< ≫)) True ≫ $
         curry∗ join ch (< ≫) $
         curry∗ wait ch (< ≫) $
         curry∗ link ch (< > •) ch
\end{code}

To test our definitions, we compose client and server in parallel
\begin{code}
Main : Proc [ 𝔹 ]
Main = curry∗ cut Client (< •) Server
\end{code}
and then ask Agda to normalize \AgdaFunction{Main}, which yields
\AgdaFunction{True} as expected.

\begin{remark}
    \label{rem:curry}
    Writing processes in Agda would be more pleasant if the constructors of the
    data type \AgdaDatatype{Proc} were naturally curried, instead of currying
    them on demand with \AgdaFunction{curry∗} as we do here.
    %
    Below is the naturally curried constructor \AgdaInductiveConstructor{fork}
    of a hypothetical data type \AgdaDatatype{Proc'}, obtained by expanding the
    definition of separating conjunction:
\begin{code}[hide]
data Proc' : Context → Set where
\end{code}
\begin{code}
  fork : ∀{A B Γ Δ Θ Θ₁ Θ₂} → Ch (A ⊗ B) Δ → Γ ≃ Δ + Θ →
         Proc' (A ∷ Θ₁) → Θ ≃ Θ₁ + Θ₂ → Proc' (B ∷ Θ₂) → Proc' (A ⊗ B ∷ Γ)
\end{code}

    This version of \AgdaInductiveConstructor{fork} is fully curried, but also
    less readable than the one we gave in \Cref{sec:process-agda} because of the
    (now visible) context splittings. We can recover some clarity and still
    obtain a curried version of \AgdaInductiveConstructor{fork} using
    (literally) the magic wand:
\begin{code}[hide]
data Proc'' : Context → Set where
\end{code}
\begin{code}
  fork : ∀{A B} →
         ∀[ Ch (A ⊗ B) ⇒ (A ∷_) ⊢ Proc'' ─∗ (B ∷_) ⊢ Proc'' ─∗ Proc'' ]
\end{code}

    However, the first arrow must be a plain implication \AgdaFunction{⇒} and
    not a magic wand to account for the appropriate amount of context
    splittings. We found this formulation of the constructors harder to explain
    and motivate in \Cref{sec:process-agda}.
    %
    Since none of the alternative definitions of \AgdaDatatype{Proc} was fully
    satisfactory, we preferred the most elegant version of the data type at the
    expense of additional clutter in this section.
    %
    \eoe
\end{remark}

\begin{code}[hide]
open import Relation.Binary.PropositionalEquality using (_≡_; refl; cong; cong₂)
\end{code}

\subsection{Type Representation}
\label{sec:type-agda}

The representation of types is standard, with one constructor for each of the
forms described in \Cref{sec:calculus}.

\begin{code}
data Type : Set where
  𝟘 𝟙 ⊥ ⊤          : Type
  ¡ ¿              : Type → Type
  _&_ _⊕_ _⊗_ _⅋_  : Type → Type → Type
\end{code}

The notion of duality is formalized as a \emph{relation} \AgdaDatatype{Dual}
such that $\AgdaDatatype{Dual}~A~B$ holds if and only if $A = \dual{B}$.

\begin{code}
data Dual : Type → Type → Set where
  d-𝟘-⊤  : Dual 𝟘 ⊤
  d-⊤-𝟘  : Dual ⊤ 𝟘
  d-𝟙-⊥  : Dual 𝟙 ⊥
  d-⊥-𝟙  : Dual ⊥ 𝟙
  d-!-?  : ∀{A B} → Dual A B → Dual (¡ A) (¿ B)
  d-?-!  : ∀{A B} → Dual A B → Dual (¿ A) (¡ B)
  d-&-⊕  : ∀{A B A′ B′} → Dual A A′ → Dual B B′ → Dual (A & B) (A′ ⊕ B′)
  d-⊕-&  : ∀{A B A′ B′} → Dual A A′ → Dual B B′ → Dual (A ⊕ B) (A′ & B′)
  d-⊗-⅋  : ∀{A B A′ B′} → Dual A A′ → Dual B B′ → Dual (A ⊗ B) (A′ ⅋ B′)
  d-⅋-⊗  : ∀{A B A′ B′} → Dual A A′ → Dual B B′ → Dual (A ⅋ B) (A′ ⊗ B′)
\end{code}

It is straightforward to prove that duality is a symmetric relation and that it
behaves as an involution. From this we prove that it acts as the function
$\dual\cdot$ defined in \Cref{sec:calculus}.

\begin{code}
dual-symm   : ∀{A B} → Dual A B → Dual B A
dual-inv    : ∀{A B C} → Dual A B → Dual B C → A ≡ C
dual-fun-r  : ∀{A B C} → Dual A B → Dual A C → B ≡ C
dual-fun-l  : ∀{A B C} → Dual B A → Dual C A → B ≡ C
\end{code}
\begin{code}[hide]
dual-symm d-𝟘-⊤ = d-⊤-𝟘
dual-symm d-⊤-𝟘 = d-𝟘-⊤
dual-symm d-𝟙-⊥ = d-⊥-𝟙
dual-symm d-⊥-𝟙 = d-𝟙-⊥
dual-symm (d-!-? p) = d-?-! (dual-symm p)
dual-symm (d-?-! p) = d-!-? (dual-symm p)
dual-symm (d-&-⊕ p q) = d-⊕-& (dual-symm p) (dual-symm q)
dual-symm (d-⊕-& p q) = d-&-⊕ (dual-symm p) (dual-symm q)
dual-symm (d-⊗-⅋ p q) = d-⅋-⊗ (dual-symm p) (dual-symm q)
dual-symm (d-⅋-⊗ p q) = d-⊗-⅋ (dual-symm p) (dual-symm q)
\end{code}

\begin{code}[hide]
dual-inv d-𝟘-⊤ d-⊤-𝟘 = refl
dual-inv d-⊤-𝟘 d-𝟘-⊤ = refl
dual-inv d-𝟙-⊥ d-⊥-𝟙 = refl
dual-inv d-⊥-𝟙 d-𝟙-⊥ = refl
dual-inv (d-!-? p) (d-?-! q) = cong ¡ (dual-inv p q)
dual-inv (d-?-! p) (d-!-? q) = cong ¿ (dual-inv p q)
dual-inv (d-&-⊕ p q) (d-⊕-& r s) = cong₂ _&_ (dual-inv p r) (dual-inv q s)
dual-inv (d-⊕-& p q) (d-&-⊕ r s) = cong₂ _⊕_ (dual-inv p r) (dual-inv q s)
dual-inv (d-⊗-⅋ p q) (d-⅋-⊗ r s) = cong₂ _⊗_ (dual-inv p r) (dual-inv q s)
dual-inv (d-⅋-⊗ p q) (d-⊗-⅋ r s) = cong₂ _⅋_ (dual-inv p r) (dual-inv q s)

dual-fun-r d e = dual-inv (dual-symm d) e

dual-fun-l d e = dual-inv d (dual-symm e)
\end{code}


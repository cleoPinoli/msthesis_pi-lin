\section{Concluding Remarks}
\label{sec:conclusion}

We have presented a formalisation of \LCP, a variant of the linear
$\pi$-calculus~\cite{KobayashiPierceTurner99} supported by the same type system
of \CP~\cite{Wadler14}. Binary sessions can be modeled in \LCP using the
continuation-passing encoding described by \citet[extended
version]{Kobayashi02b} and \citet{DardhaGiachinoSangiorgi17}.
%
Under the assumption that the size of a formalisation is a good measure of its
complexity, the formalisation of \LCP can be qualified as ``simple'' compared to
the existing ones.
%
In this respect, our work casts some doubts on the actual contribution of
context splitting to the complexity of a formalisation and hints instead at
explicit continuations in the underlying model as a simplifying factor.
%
It is quite interesting to note that, among the approaches that have been
proposed to overcome context splitting, the one of \citet{SanoKavanaghPientka23}
makes use of explicit continuations.

We remark that \LCP and the calculus described in the linearity
challenge~\cite{CarboneEtAl24} are fundamentally incomparable. On the one hand,
the challenge only considers a minimal calculus of first-order, monomorphic
sessions while \LCP is a fully-fledged calculus that supports linear, shared,
higher-order, polymorphic channels. On the other hand, the logical foundations
of \LCP prevent the modeling of deadlocks and of sequential processes that own
both endpoints of a session, both of which can be modeled in the calculus of the
challenge.
%
Because of its logical foundations, we like to think of \LCP as a ``canonical''
calculus that deserves its own space in the context of the linearity challenge
alongside with (but not in substitution of) more traditional session calculi.

The simple formalisation of \LCP is a good basis for further investigations. We
have already extended \LCP with support for infinite types and recursive
processes (the formalisation of this extension is publicly available in a
sibling folder of the one for \LCP). In the future, it would be interesting to
formalise the strong normalisation property of \LCP as a consequence of cut
elimination of classical linear logic.

In this work we have focused on models and theories of \emph{binary sessions}
(those connecting exactly two processes), but there are also formalisations of
\emph{multiparty sessions}, notably those of \citet{JacobsBalzerKrebbers22a} and
\citet{TiroreBengstonCarbone25}, which are significantly more complex than those
of binary sessions. Just to give an example, the formalisation by
\citet{TiroreBengstonCarbone25} amounts to more than 1Mb of Coq code.
Incidentally, multiparty sessions are not supported by a strong connection with
linear logic and there is no known continuation-passing encoding of multiparty
sessions. Hence, it is unclear whether the considerations made for this work
also apply to the multiparty case.
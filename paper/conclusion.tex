\section{Concluding Remarks}
\label{sec:conclusion}

We have presented a formalisation of \LCC, a linear calculus of continuations
closely related to the linear $\pi$-calculus~\cite{KobayashiPierceTurner99} and
supported by the same type system of \CP~\cite{Wadler14}. Binary sessions can be
modeled in \LCC using the continuation-passing encoding described by
\citet[extended version]{Kobayashi02b} and \citet{DardhaGiachinoSangiorgi17}.

The linear calculus of continuations and the calculus described in the linearity
challenge~\cite{CarboneEtAl24} are incomparable in terms of expressiveness. The
challenge only considers a minimal calculus of first-order, monomorphic sessions
while \LCC supports linear, shared, higher-order, polymorphic channels. At the
same time, the logical foundations of \LCC prevent the modeling of deadlocked
processes, of sequential processes that own both endpoints of a session and in
general of cyclic network topologies. Such processes can be modeled in the
calculus of the challenge, where parallel composition and name restriction are
distinct constructs as in the $\pi$-calculus.
%
Because of its logical foundations, we think that \LCC deserves its own space in
the context of the linearity challenge alongside with (but not in substitution
of) more traditional session calculi.

Considering the richness of \LCC in terms of features and proved properties, the
small (and possibly simple) formalisation of \LCC casts some doubts on the
actual role of context splitting as a source of unnecessary complexity. We
perceive far greater benefits from the adoption of a model with explicit
continuations and we find hints of this also beyond our work.
%
Among the alternative approaches that have been proposed to overcome the
difficulties of context splitting, the one of \citet{SanoKavanaghPientka23}
makes use of explicit continuations.
%
In a different setting, the continuation-based encoding of binary sessions has
greatly simplified the problem of session type reconstruction~\cite{Padovani17}.

The compact formalisation of \LCC is a good starting point for further
developments. We have already extended \LCC with support for coinductive (\ie
possibly infinite) types and recursive processes (the source code of this
extension is publicly available in the same repository of
\LCC~\cite{PadovaniRaffaelli25}). In the future, it would be interesting to
formalise the strong normalisation property of \LCC as a consequence of cut
elimination of classical linear logic.

In this work we have focused on models of \emph{binary sessions} (those
connecting exactly two processes), but there are also formalisations of
\emph{multiparty sessions}, notably those by \citet{JacobsBalzerKrebbers22a} and
\citet{TiroreBengstonCarbone25}, which can be significantly more complex than
those of binary sessions. The formalisation by \citet{JacobsBalzerKrebbers22a}
amounts to 173kb and the one by \citet{TiroreBengstonCarbone25} to more than 1Mb
of Coq code. Also in these cases, the formalisation based on (virtual)
continuations~\citep{JacobsBalzerKrebbers22a} happens to be substantially
smaller. Whether this is a coincidence or further evidence of the effectiveness
of the continuation-based approaches is left as future work.

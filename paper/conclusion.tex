\section{Concluding Remarks}
\label{sec:conclusion}

We have presented a formalisation of \LCC, a linear calculus of continuations
closely related to the linear $\pi$-calculus~\cite{KobayashiPierceTurner99} and
supported by the same type system of \CP~\cite{Wadler14}. Binary sessions can be
modeled in \LCC using the continuation-passing encoding described by
\citet[extended version]{Kobayashi02b} and \citet{DardhaGiachinoSangiorgi17}.

The linear calculus of continuations and the calculus described in the linearity
challenge~\cite{CarboneEtAl24} are incomparable in terms of expressiveness. On
the one hand, the challenge only considers a minimal calculus of first-order,
monomorphic sessions while \LCC supports linear, shared, higher-order,
polymorphic channels; on the other hand, the calculus of the challenge allows
the modeling of sequential processes owning both endpoints of a session and in
general of cyclic network topologies, none of which can be modeled in \LCC
because of its tight correspondence with linear logic.
%
We think that \LCC deserves its own space in the context of the linearity
challenge alongside with (but not in substitution of) more traditional session
calculi.

Considering the richness of \LCC in terms of features and proved properties, the
simple formalisation of \LCC casts some doubts on the actual role of context
splitting as a source of complexity. We perceive more tangible benefits from the
adoption of a calculus with explicit continuations where channels are linear in
a literal sense.
%
In this respect, we find it intriguing that, among the alternative approaches
that have been proposed to overcome the difficulties of context splitting, the
one by \citet{SanoKavanaghPientka23} makes key use of explicit continuations.

The compact formalisation of \LCC is a good starting point for further
developments. We have already extended \LCC with support for coinductive (\ie
possibly infinite) types and recursive processes (this extension is in \LCC's
public repository~\cite{PadovaniRaffaelli25}). In the future, it would be
interesting to formalise the strong normalisation property of \LCC as a
consequence of cut elimination of classical linear logic.

In this work we have focused on models of \emph{binary sessions} (those
connecting exactly two processes), but there are also formalisations of
\emph{multiparty sessions}, notably those by \citet{JacobsBalzerKrebbers22a} and
\citet{TiroreBengstonCarbone25}, which can be significantly more complex than
those of binary sessions. The formalisation by \citet{JacobsBalzerKrebbers22a}
amounts to 173kb and the one by \citet{TiroreBengstonCarbone25} to more than 1Mb
of Coq code. Also in these cases, the formalisation based on (virtual)
continuations~\citep{JacobsBalzerKrebbers22a} happens to be substantially
smaller. Whether this is a coincidence or further evidence of the effectiveness
of the continuation-based approaches is left for future investigations.

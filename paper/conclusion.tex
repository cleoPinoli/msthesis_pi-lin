\section{Concluding Remarks}
\label{sec:conclusion}

We have presented a formalisation of \LCP, a variant of the linear
$\pi$-calculus~\cite{KobayashiPierceTurner99} supported by the same type system
of \CP~\cite{Wadler14}. Binary sessions are modeled in \LCP using the
continuation-passing encoding described by \citet[extended
version]{Kobayashi02b} and \citet{DardhaGiachinoSangiorgi17}.
%
Under the assumption that the size of a formalisation is a good measure of its
complexity, the formalisation of \LCP can be qualified as ``simple'' compared to
the existing ones.
%
In this respect, our work casts some doubts on the actual contribution of
context splitting to the complexity of a formalisation and hints instead at
explicit continuations in the underlying model as a simplifying factor.
%
It is quite striking that, among the approaches that have been proposed to
overcome context splitting, the one of \citet{SanoKavanaghPientka23} makes use
of explicit continuations.

We remark that \LCP and the calculus described in the linearity
challenge~\cite{CarboneEtAl24} are incomparable in terms of expressiveness. On
the one hand, the challenge only considers a minimal calculus of first-order,
monomorphic sessions while \LCP is a fully-fledged calculus that supports
linear, shared, higher-order, polymorphic channels. On the other hand, the
logical foundations of \LCP prevent the modeling of deadlocked processes and of
sequential processes that own both endpoints of a session. These processes can
be modeled in the calculus of the challenge, where parallel composition and name
restriction are distinct constructs as in the $\pi$-calculus.
%
Because of its logical foundations, we like to think of \LCP as a ``canonical''
calculus that deserves its own space in the context of the linearity challenge
alongside with (but not in substitution of) more traditional session calculi.

The simple formalisation of \LCP is a good basis for further investigations. We
have already extended \LCP with support for infinite types and recursive
processes (the source code of this extension is publicly available in the same
repository as the one that contains the formalisation of \LCP). In the future,
it would be interesting to formalise the strong normalisation property of \LCP
as a consequence of cut elimination of classical linear logic.

In this work we have focused on models and theories of \emph{binary sessions}
(those connecting exactly two processes), but there are also formalisations of
\emph{multiparty sessions}, notably those by \citet{JacobsBalzerKrebbers22a} and
\citet{TiroreBengstonCarbone25}, which can be significantly more complex than
those of binary sessions. The formalisation by \citet{JacobsBalzerKrebbers22a}
amounts to 173kb and the one by \citet{TiroreBengstonCarbone25} to more than 1Mb
of Coq code. Also in these cases, the formalisation based on (virtual)
continuations~\citep{JacobsBalzerKrebbers22a} happens to be substantially
smaller. Whether this is a coincidence or further evidence of the effectiveness
of the continuation-based approaches is left for future work.

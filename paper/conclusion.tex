\section{Concluding Remarks}
\label{sec:conclusion}

We have presented a formalisation of \LCC, a linear calculus of continuations
closely related to the linear $\pi$-calculus~\cite{KobayashiPierceTurner99} and
supported by the same type system of \CP~\cite{Wadler14}. Binary sessions can be
modeled in \LCC using the continuation-passing encoding described by
\citet[extended version]{Kobayashi02b} and \citet{DardhaGiachinoSangiorgi17}.
%
Under the assumption that the size of a formalisation is a good measure of its
complexity, the formalisation of \LCC qualifies as the simplest compared to the
existing ones. This result should also be weighed considering the richness of
\LCC in terms of supported features and provable properties.

Our work casts some doubts on the actual contribution of context splitting to
the complexity of a formalisation and hints instead at explicit continuations in
the underlying model as a simplifying factor.
%
It is quite remarkable that, among the alternative approaches that have been
proposed to avoid context splitting, the one of \citet{SanoKavanaghPientka23}
based on linearity predicates makes use of explicit continuations.

The calculus of linear continuations and the calculus described in the linearity
challenge~\cite{CarboneEtAl24} are incomparable in terms of expressiveness. On
the one hand, the challenge only considers a minimal calculus of first-order,
monomorphic sessions while \LCC is a fully-fledged calculus that supports
linear, shared, higher-order, polymorphic channels. On the other hand, the
logical foundations of \LCC prevent the modeling of deadlocked processes and of
sequential processes that own both endpoints of a session. These processes can
be modeled in the calculus of the challenge, where parallel composition and name
restriction are distinct constructs as in the $\pi$-calculus.
%
Because of its logical foundations, we think that \LCC deserves its own space in
the context of the linearity challenge alongside with (but not in substitution
of) more traditional session calculi.

The simple formalisation of \LCC is a good starting point for further
developments. We have already extended \LCC with support for infinite types and
recursive processes (the source code of this extension is publicly available in
the same repository as the one that contains the formalisation of \LCC). In the
future, it would be interesting to formalise the strong normalisation property
of \LCC as a consequence of cut elimination of classical linear logic.

In this work we have focused on models and theories of \emph{binary sessions}
(those connecting exactly two processes), but there are also formalisations of
\emph{multiparty sessions}, notably those by \citet{JacobsBalzerKrebbers22a} and
\citet{TiroreBengstonCarbone25}, which can be significantly more complex than
those of binary sessions. The formalisation by \citet{JacobsBalzerKrebbers22a}
amounts to 173kb and the one by \citet{TiroreBengstonCarbone25} to more than 1Mb
of Coq code. Also in these cases, the formalisation based on (virtual)
continuations~\citep{JacobsBalzerKrebbers22a} happens to be substantially
smaller. Whether this is a coincidence or further evidence of the effectiveness
of the continuation-based approaches is left as future work.

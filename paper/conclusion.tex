\section{Concluding Remarks}
\label{sec:conclusion}

In this work we have focused on models and theories of \emph{binary sessions},
but there are also formalizations of \emph{multiparty session types}, notably
those of \citet{JacobsBalzerKrebbers22a} and \citet{TiroreBengstonCarbone25},
which are notoriously more complex than those of binary session types. Just to
give an example, the formalization of \citet{TiroreBengstonCarbone25} amounts to
more than 1Mb of Coq code.

FARE CONFRONTO TRA LCP E IL CALCOLO NELLA CHALLENGE PER DIRE CHE NON SI RIESCONO
A MODELLARE TUTTI I PROCESSI, MAGARI SI PUÒ FARE UN ESEMPIO NELLA SEZIONE SUL
CALCOLO.

PARLARE DI CP E DI PI CALCOLO LINEARE

BELLIN E SCOTT?

context splitting hard to read but overall doable

working with an intrinsically typed calculus drastically reduces the number of
cases that must be considered in proving properties (eg deadlock freedom)

it also makes it trivial to reason about properties of sub-processes (see type
safety, where Q is well typed)

structural pre-congruence by far most involved, and it will grow even further in
order to prove more complex properties (eg confluence)

linear pi-calculus is good

we think that a simple formalization is a good basis for further investigations.
For examples, defined cut elimination, strong normalization, soundness of
subtyping, for the linear fragment of \Calculus

conclusioni: comparison in \Cref{sec:related} made under the assumption that the
size of the formalization is a good measure of its complexity. But there are
several factors that may invalidate this assumption. We should pursue 

%Version 3 October 2023
% See section 11 of the User Manual for version history
%
%%%%%%%%%%%%%%%%%%%%%%%%%%%%%%%%%%%%%%%%%%%%%%%%%%%%%%%%%%%%%%%%%%%%%%
%%                                                                 %%
%% Please do not use \input{...} to include other tex files.       %%
%% Submit your LaTeX manuscript as one .tex document.              %%
%%                                                                 %%
%% All additional figures and files should be attached             %%
%% separately and not embedded in the \TeX\ document itself.       %%
%%                                                                 %%
%%%%%%%%%%%%%%%%%%%%%%%%%%%%%%%%%%%%%%%%%%%%%%%%%%%%%%%%%%%%%%%%%%%%%

%%\documentclass[referee,sn-basic]{sn-jnl}% referee option is meant for double line spacing

%%=======================================================%%
%% to print line numbers in the margin use lineno option %%
%%=======================================================%%

%%\documentclass[lineno,sn-basic]{sn-jnl}% Basic Springer Nature Reference Style/Chemistry Reference Style

%%======================================================%%
%% to compile with pdflatex/xelatex use pdflatex option %%
%%======================================================%%

%%\documentclass[pdflatex,sn-basic]{sn-jnl}% Basic Springer Nature Reference Style/Chemistry Reference Style


%%Note: the following reference styles support Namedate and Numbered referencing. By default the style follows the most common style. To switch between the options you can add or remove �Numbered� in the optional parenthesis. 
%%The option is available for: sn-basic.bst, sn-vancouver.bst, sn-chicago.bst%  
 
%%\documentclass[sn-nature]{sn-jnl}% Style for submissions to Nature Portfolio journals
%%\documentclass[sn-basic]{sn-jnl}% Basic Springer Nature Reference Style/Chemistry Reference Style
\documentclass[sn-mathphys-num]{sn-jnl}% Math and Physical Sciences Numbered Reference Style 
%%\documentclass[sn-mathphys-ay]{sn-jnl}% Math and Physical Sciences Author Year Reference Style
%%\documentclass[sn-aps]{sn-jnl}% American Physical Society (APS) Reference Style
%%\documentclass[sn-vancouver,Numbered]{sn-jnl}% Vancouver Reference Style
%%\documentclass[sn-apa]{sn-jnl}% APA Reference Style 
%%\documentclass[sn-chicago]{sn-jnl}% Chicago-based Humanities Reference Style

%%%% Standard Packages
%%<additional latex packages if required can be included here>

\usepackage{amsmath,amssymb,amsfonts}
\usepackage{amsthm}
%\usepackage{mathrsfs}
\usepackage[title]{appendix}
\usepackage{xcolor}
\usepackage{textcomp}
%\usepackage{manyfoot}
\usepackage{listings}
\usepackage{cleveref}
\usepackage{mathtools}
\usepackage{stmaryrd}
\usepackage{mathpartir}
\usepackage{agda}
\usepackage{xspace}
\usepackage{upgreek}
%%%%

%%%%%%%%%%%%%%%%%%%%
%%% CONDITIONALS %%%
%%%%%%%%%%%%%%%%%%%%

\newif\ifcomments
\commentstrue
%\commentsfalse

\newif\ifslotserver
\slotservertrue

%%%%%%%%%%%%%%%%
%%% NOTATION %%%
%%%%%%%%%%%%%%%%

\newcommand{\Calculus}{$\pi$\textsf{LIN}\xspace}

\newcommand{\eg}{\textit{e.g.}\xspace}
\newcommand{\ie}{\textit{i.e.}\xspace}
\newcommand{\cf}{\textit{cf.}\xspace}

\newcommand{\mktag}[1]{\mathsf{\color{teal}#1}}
\newcommand{\mkkeyword}[1]{\mathsf{\color{blue}#1}}
\newcommand{\mkfunction}[1]{\mathsf{#1}}

\newcommand{\parens}[1]{(#1)}
\newcommand{\braces}[1]{\{#1\}}
\newcommand{\bracks}[1]{[#1]}
\newcommand{\angles}[1]{\langle#1\rangle}

\newcommand{\set}[1]{\braces{#1}}
\newcommand{\seqof}[1]{\overline{#1}}
\newcommand{\emptybag}{\bag\!}
\newcommand{\bag}[1]{\mathopen\lbag#1\mathclose\rbag}
\newcommand{\bigbag}[1]{%
  \lbag\begin{array}[t]{@{}l@{}}
    #1 \rbag
  \end{array}
}

\newcommand{\rulename}[1]{\textnormal{\textsc{\small[#1]}}\xspace}

\newcommand{\defrule}[2][]{\hypertarget{rule:\ifblank{#1}{#2}{#1}}{\rulename{#2}}}
%\newcommand{\refrule}[2][]{\hyperlink{rule:\ifblank{#1}{#2}{#1}}{\rulename{#2}}}
\newcommand{\refrule}[2][]{\rulename{#2}\xspace}

\newcommand{\infercorule}[3][]{%
  {\mprset{fraction={===}}\inferrule[#1]{#2}{#3}}%
}

% \newcommand{\defrule}[2][]{\rulename{#1}}
% \newcommand{\refrule}[2][]{\rulename{#1}}

\newcommand{\proofcase}[1]{\textbf{Case #1.}\xspace}

\newcommand{\Nat}{\mathbb{N}}
\newcommand{\Real}{\mathbb{R}}
\newcommand{\RealP}{\Real_{\geq0}}
\newcommand{\CSLL}{\textsf{CSLL}\xspace}
\newcommand{\CoreCSLL}{\texorpdfstring{$\textsf{\upshape CSLL}^\infty$}{coreCSLL}\xspace}
\newcommand{\isdet}{\textsf{\upshape det}}
\newcommand{\DetCSLL}{$\textsf{\upshape CSLL}^\infty_{\isdet}$\xspace}
\newcommand{\muMALL}{\texorpdfstring{$\mu\textsf{\upshape MALL}^\infty$}{muMALL}\xspace}
\newcommand{\piLIN}{\texorpdfstring{$\pi\textsf{LIN}$}{piLIN}\xspace}
\newcommand{\CP}{$\textsf{CP}$\xspace}
\newcommand{\muCP}{$\mu\textsf{CP}$\xspace}
\newcommand{\piDILL}{$\pi\textsf{DILL}$\xspace}
\newcommand{\SILLS}{$\textsf{SILL}_{\textsf{S}}$\xspace}
\newcommand{\HCPND}{$\textsf{HCP}_{\textsf{ND}}$\xspace}
\newcommand{\picalculus}{\texorpdfstring{$\pi$-calculus}{pi-calculus}\xspace}

\newcommand\eoe{\hfill$\lrcorner$}

\setlength{\marginparsep}{5mm}
\setlength{\marginparwidth}{2cm}
\newcommand{\marginnote}[2]{%
  \ifcomments%
    $^{\color{magenta}\mathclap\star}$%
    \marginpar[
        \flushright\tiny\sf\textbf{#1}: #2
    ]{
        \flushleft\tiny\sf\textbf{#1}: #2
    }%
  \fi%
}
\newcommand{\Luca}[1]{\marginnote{Luca}{\color{blue}#1}}
\newcommand{\Ugo}[1]{\marginnote{Ugo}{\color{red}#1}}
\newcommand{\BEGINREVISION}{\color{magenta}}
\newcommand{\REVISION}[1]{{\color{magenta}#1}}
%\newcommand{\REVISION}[1]{#1}

\newcommand{\TODO}{\textsf{\textbf{\color{red}TODO}}\xspace}

%%%%%%%%%%%%%%%%%
%%% PROCESSES %%%
%%%%%%%%%%%%%%%%%

\newcommand\A{\mathsf{A}}
\newcommand\B{\mathsf{B}}
\newcommand\C{\mathsf{C}}
\newcommand\D{\mathsf{D}}

\newcommand\Run{\rho}

\newcommand{\VarSet}{\mathcal{V}}
\newcommand{\ProcessSet}{\mathcal{P}}

\newcommand{\InTag}{\mktag{in}}
\newcommand{\LeftTag}{\InTag_1}
\newcommand{\RightTag}{\InTag_2}
\newcommand{\AddTag}{\mktag{add}}
\newcommand{\PayTag}{\mktag{pay}}
\newcommand{\PlayTag}{\mktag{play}}
\newcommand{\QuitTag}{\mktag{quit}}
\newcommand{\WinTag}{\mktag{win}}
\newcommand{\LoseTag}{\mktag{lose}}

\newcommand{\x}{x}
\newcommand{\y}{y}
\newcommand{\z}{z}
\renewcommand{\u}{u}
\renewcommand{\v}{v}

\newcommand{\out}[1]{#1}
\newcommand{\inp}[1]{#1}
\newcommand{\parop}{\mathbin|}

\newcommand{\Def}[2]{#1\ifblank{#2}{}{\parens{#2}}}
\newcommand{\Call}[2]{#1\ifblank{#2}{}{\angles{#2}}}
\newcommand{\Link}[2]{#1\leftrightarrow#2}
\newcommand{\Close}[1]{\mkkeyword{close}\,#1}
\newcommand{\Wait}[1]{\mkkeyword{wait}\,#1}
\newcommand{\Fail}[1]{\mkkeyword{fail}\,#1}
\newcommand{\Select}[3]{#1\bracks{#2}\ifblank{#3}{}{\angles{#3}}}
\newcommand{\Left}[2]{\Select{#1}{\InTag_1}{#2}}
\newcommand{\Right}[2]{\Select{#1}{\InTag_2}{#2}}
\newcommand{\Choice}[3][p]{#2 \mathbin{\ifblank{#1}{}{_{#1}}\boxplus} #3}
\newcommand{\OneCase}[3]{#1\ifblank{#2}{}{\parens{#2}}#3}
\newcommand{\Case}[4]{\OneCase{#1}{#2}{\braces{#3,#4}}}
\newcommand{\BigCase}[2]{\OneCase{#1}{\left\{\begin{array}{@{}r@{~:~}l@{}}#2\end{array}\right\}}}
\newcommand{\Fork}[5]{#1\angles{#2\ifblank{#3}{}{,#3}}\parens{#4,#5}}
\renewcommand{\Join}[3]{#1\parens{#2\ifblank{#3}{}{,#3}}}
\newcommand{\Client}[2]{{?}#1\bracks{#2}}
\newcommand{\Server}[2]{{!}#1\parens{#2}}
\newcommand{\Cut}[3]{\parens{#1}\parens{#2\parop#3}}
\newcommand{\BigCut}[3]{\parens{#1}\left(#2\,\middle|\,#3\right)}

\newcommand{\Hole}{\bracks\,}
\newcommand{\RC}{\mathcal{C}}
\newcommand{\RD}{\mathcal{D}}

\newcommand{\Dist}{\DistM}
\newcommand{\DistM}{M}
\newcommand{\DistN}{N}

%%%%%%%%%%%%%%%%%%%%%%%%%%
%%% FORMULAS AND TYPES %%%
%%%%%%%%%%%%%%%%%%%%%%%%%%

\newcommand{\Type}{\TypeT}
\newcommand{\TypeT}{T}
\newcommand{\TypeS}{S}

\newcommand{\X}{X}
\newcommand{\Y}{Y}

\newcommand{\Bot}{\bot}
\newcommand{\Top}{\top}
\newcommand{\One}{\mathbf{1}}
\newcommand{\Zero}{\mathbf{0}}
\newcommand{\Plus}{\mathbin\oplus}
\newcommand{\With}{\mathbin\binampersand}
\newcommand{\Ten}{\mathbin\otimes}
\newcommand{\Par}{\mathbin\bindnasrepma}
\newcommand{\OfCourse}[1][]{\ifblank{#1}{{!}}{\ann{{!}}{#1}}}
\newcommand{\WhyNot}[1][]{\ifblank{#1}{{?}}{\ann{{?}}{#1}}}

\newcommand{\ann}[2]{#1^{{\color{purple}#2}}}

\newcommand{\Sequent}{\SequentS}
\newcommand{\SequentS}{\Sigma}
\newcommand{\SequentT}{\Theta}

\newcommand{\SComm}{\rulename{s-comm}}
\newcommand{\SLink}{\rulename{s-link}}
\newcommand{\SWait}{\rulename{s-wait}}
\newcommand{\SSelect}{\rulename{s-select}}
\newcommand{\SFail}{\rulename{s-fail}}
\newcommand{\SCase}{\rulename{s-case}}
\newcommand{\SForkL}{\rulename{s-fork-l}}
\newcommand{\SForkR}{\rulename{s-fork-r}}
\newcommand{\SJoin}{\rulename{s-join}}

\newcommand{\RLink}{\rulename{r-link}}
\newcommand{\RFail}{\rulename{r-fail}}
\newcommand{\RClose}{\rulename{r-close}}
\newcommand{\RFork}{\rulename{r-fork}}
\newcommand{\RSelect}{\rulename{r-select}}
\newcommand{\RCut}{\rulename{r-cut}}
\newcommand{\RCong}{\rulename{r-cong}}

\newcommand{\RDrop}{\rulename{r-drop}}
\newcommand{\RStep}{\rulename{r-step}}
\newcommand{\RComb}{\rulename{r-comb}}

%%%%%%%%%%%%%%%%%%%
%%% TYPE SYSTEM %%%
%%%%%%%%%%%%%%%%%%%

\newcommand{\EmptyContext}{\emptyset}
\newcommand{\Context}{\ContextC}
\newcommand{\ContextC}{\Upgamma}
\newcommand{\ContextD}{\Updelta}
\newcommand{\un}{{?}}

\newcommand{\Tag}[1][a]{\mathsf{\color{teal}#1}}

\newcommand{\wtp}[3][]{#2 \vdash\ifblank{#1}{}{^{\color{purple}#1}} #3}

\newcommand{\CallRule}{\rulename{call}}
\newcommand{\LinkRule}{\rulename{link}}
\newcommand{\CutRule}{\rulename{cut}}
\newcommand{\FailRule}{\rulename{$\Top$}}
\newcommand{\CloseRule}{\rulename{$\One$}}
\newcommand{\WaitRule}{\rulename{$\Bot$}}
\newcommand{\ForkRule}{\rulename{$\Ten$}}
\newcommand{\JoinRule}{\rulename{$\Par$}}
\newcommand{\SelectRule}{\rulename{$\Plus$}}
\newcommand{\CaseRule}{\rulename{$\With$}}
\newcommand{\ServerRule}{\rulename{server}}
\newcommand{\ClientRule}{\rulename{client}}
\newcommand{\WeakRule}{\rulename{weak}}
\newcommand{\ChoiceRule}{\rulename{$\Choice[]{}{}$}}
\newcommand{\ChoiceRuleAST}{\rulename{$\Choice[]{}{}$-ast}}
\newcommand{\DistRule}{\rulename{distribution}}

\newcommand{\Length}{\ell}

%%%%%%%%%%%%%%%%%
%%% RELATIONS %%%
%%%%%%%%%%%%%%%%%

\newcommand{\pcong}{\sqsupseteq}
\newcommand{\red}{\rightarrow}
\newcommand{\reds}{\Rightarrow}
\newcommand{\wred}{\Rightarrow}
\newcommand{\nred}{\arrownot\red}
\newcommand{\eqdef}{\stackrel{\smash{\textsf{\upshape\tiny def}}}=}

\newcommand{\prefix}{\sqsubseteq}
\newcommand{\subf}{\preceq}
\newcommand{\tred}{\leadsto}
\newcommand{\rtred}{\tred^?}

%%%%%%%%%%%%%%%%%
%%% FUNCTIONS %%%
%%%%%%%%%%%%%%%%%

\newcommand{\fn}[1]{\mkfunction{fn}\parens{#1}}
\newcommand{\bn}[1]{\mkfunction{bn}\parens{#1}}
\newcommand{\subst}[2]{\braces{#1/#2}}
\newcommand{\dual}[1]{#1^{\bot}}
\newcommand{\dom}[1]{\mkfunction{dom}\parens{#1}}
\newcommand{\InfOften}[1]{\mkfunction{inf}\parens{#1}}
\newcommand{\minf}{\mkfunction{min}\,}
% \newcommand{\strip}[1]{\overline{#1}}
\newcommand{\depth}[1]{\mathsf{depth}\parens{#1}}
\newcommand{\rankof}[1]{|#1|}
\newcommand{\fc}[2]{\mathsf{fc}(#1,#2)}
\newcommand{\dsup}{\sqcup}
\newcommand{\sem}[1]{\llbracket#1\rrbracket}

\newcommand{\prob}[1]{|#1|}
\newcommand{\trunc}[1]{\lfloor#1\rfloor}
\newcommand{\edl}[1]{\mathsf{edl}\parens{#1}}
\newcommand{\edh}[1]{\mathsf{edh}\parens{#1}}

%%%%%%%%%%%%%%%%%%%%
%%% ENVIRONMENTS %%%
%%%%%%%%%%%%%%%%%%%%

% \theoremstyle{plain}
% \newtheorem{theorem}{Theorem}[section]
% \newtheorem{lemma}{Lemma}[section]
% \newtheorem{proposition}{Proposition}[section]
% \newtheorem{corollary}{Corollary}[section]

% \theoremstyle{definition}
% \newtheorem{definition}{Definition}[section]
% \newtheorem{example}{Example}[section]

% \theoremstyle{remark}
% \newtheorem{remark}{Remark}[section]

% \theoremstyle{remark}
% \newtheorem{notation}[theorem]{Notation}

%%% Local Variables:
%%% mode: latex
%%% TeX-master: "main"
%%% End:


%%%%%=============================================================================%%%%
%%%%  Remarks: This template is provided to aid authors with the preparation
%%%%  of original research articles intended for submission to journals published 
%%%%  by Springer Nature. The guidance has been prepared in partnership with 
%%%%  production teams to conform to Springer Nature technical requirements. 
%%%%  Editorial and presentation requirements differ among journal portfolios and 
%%%%  research disciplines. You may find sections in this template are irrelevant 
%%%%  to your work and are empowered to omit any such section if allowed by the 
%%%%  journal you intend to submit to. The submission guidelines and policies 
%%%%  of the journal take precedence. A detailed User Manual is available in the 
%%%%  template package for technical guidance.
%%%%%=============================================================================%%%%

%% as per the requirement new theorem styles can be included as shown below
\theoremstyle{thmstyleone}%
\newtheorem{theorem}{Theorem}%  meant for continuous numbers
%%\newtheorem{theorem}{Theorem}[section]% meant for sectionwise numbers
%% optional argument [theorem] produces theorem numbering sequence instead of independent numbers for Proposition
\newtheorem{proposition}[theorem]{Proposition}% 
%%\newtheorem{proposition}{Proposition}% to get separate numbers for theorem and proposition etc.

\theoremstyle{thmstyletwo}%
\newtheorem{example}{Example}%
\newtheorem{remark}{Remark}%

\theoremstyle{thmstylethree}%
\newtheorem{definition}{Definition}%

\raggedbottom
%%\unnumbered% uncomment this for unnumbered level heads

\begin{document}

\title{A Logical Solution of the Linearity Challenge}

%%=============================================================%%
%% GivenName	-> \fnm{Joergen W.}
%% Particle	-> \spfx{van der} -> surname prefix
%% FamilyName	-> \sur{Ploeg}
%% Suffix	-> \sfx{IV}
%% \author*[1,2]{\fnm{Joergen W.} \spfx{van der} \sur{Ploeg} 
%%  \sfx{IV}}\email{iauthor@gmail.com}
%%=============================================================%%

\author*[1]{\fnm{Luca} \sur{Padovani}}\email{luca.padovani2@unibo.it}

\author[2]{\fnm{Claudia} \sur{Raffaelli}}\email{xxx@yyy.zzz}
\equalcont{These authors contributed equally to this work.}

\affil*[1]{\orgdiv{Department of Computer Science and Engineering}, \orgname{Università di Bologna}, \orgaddress{\street{Mura Anteo Zamboni 7}, \city{Bologna}, \postcode{40138}, \state{BO}, \country{Italy}}}

\affil[2]{\orgdiv{Computer Science Division}, \orgname{Università di Camerino}, \orgaddress{\street{via Madonna delle Carceri 7}, \city{Camerino}, \postcode{62032}, \state{MC}, \country{Italy}}}

%%==================================%%
%% Sample for unstructured abstract %%
%%==================================%%

\abstract{blah}

%%================================%%
%% Sample for structured abstract %%
%%================================%%

% \abstract{\textbf{Purpose:} The abstract serves both as a general introduction to the topic and as a brief, non-technical summary of the main results and their implications. The abstract must not include subheadings (unless expressly permitted in the journal's Instructions to Authors), equations or citations. As a guide the abstract should not exceed 200 words. Most journals do not set a hard limit however authors are advised to check the author instructions for the journal they are submitting to.
% 
% \textbf{Methods:} The abstract serves both as a general introduction to the topic and as a brief, non-technical summary of the main results and their implications. The abstract must not include subheadings (unless expressly permitted in the journal's Instructions to Authors), equations or citations. As a guide the abstract should not exceed 200 words. Most journals do not set a hard limit however authors are advised to check the author instructions for the journal they are submitting to.
% 
% \textbf{Results:} The abstract serves both as a general introduction to the topic and as a brief, non-technical summary of the main results and their implications. The abstract must not include subheadings (unless expressly permitted in the journal's Instructions to Authors), equations or citations. As a guide the abstract should not exceed 200 words. Most journals do not set a hard limit however authors are advised to check the author instructions for the journal they are submitting to.
% 
% \textbf{Conclusion:} The abstract serves both as a general introduction to the topic and as a brief, non-technical summary of the main results and their implications. The abstract must not include subheadings (unless expressly permitted in the journal's Instructions to Authors), equations or citations. As a guide the abstract should not exceed 200 words. Most journals do not set a hard limit however authors are advised to check the author instructions for the journal they are submitting to.}

\keywords{keyword1, Keyword2, Keyword3, Keyword4}

%%\pacs[JEL Classification]{D8, H51}

%%\pacs[MSC Classification]{35A01, 65L10, 65L12, 65L20, 65L70}

\maketitle

\section{Introduction}
\label{sec:introduction}

\section{\Calculus}
\label{sec:calculus}

\subsection{Syntax}
\label{sec:syntax}

\begin{table}
    \caption{Syntax of \Calculus.}\label{tab:syntax}
    \begin{tabular}{c}
        \begin{math}
            \displaystyle
            \begin{array}[t]{rcll}
                P, Q & ::= & \Link\x\y            & \text{link} \\
                    & | & \Fail\x                 & \text{fail} \\
                    & | & \Close\x                & \text{close} \\
                    & | & \Wait\x.P               & \text{wait} \\
                    & | & \Select\x{\InTag_i}\z.P & \text{select} \\
                    & | & \Case\x\z{P}{Q}         & \text{branch} \\
                    & | & \Fork\x\y\z{P}{Q}       & \text{fork} \\
                    & | & \Join\x\y\z.P           & \text{join} \\
                    & | & \Server\x\y.P           & \text{accept} \\
                    & | & \Client\x\y.P           & \text{request} \\
                    & | & \Weaken\x.P             & \text{weaken} \\
                    & | & \Contract\x\y\z.P       & \text{contract} \\
                    & | & \Cut\x{P}{Q}            & \text{cut} \\
            \end{array}
        \end{math}
    \end{tabular}
\end{table}

The calculus of pure sessions we formalized is dubbed \Calculus and is a linear
version of \CP~\citep{Wadler14,GayVasconcelos25} such that every session
endpoint is meant to be used \emph{exactly once}. This differs from \CP and
other session calculi where session endpoints can be used repeatedly, albeit
sequentially. The key idea is that each message sent on a session endpoint is
paired with a \emph{fresh continuation channel} on which the rest of the
conversation takes place.
%
This way of encoding sessions using explicit continuations has been first
described by \citet{Kobayashi} and later studied extensively by
\citet{DardhaGiachinoSangiorgi17}.

The syntax of \Calculus makes use of an infinite set of \emph{channel names},
ranged over by $x$, $y$ and $z$, and is shown in~\Cref{tab:syntax}.
%
A link $\Link\x\y$ denotes the merging of the channels $x$ and $y$, so that
every message sent on one of the channels is forwarded to the other. As
discussed in the literature, this form is useful for modeling \emph{delegation},
that is the transmission of an existing channel over another channel.
%
The $\Close\x$ and $\Wait\x.P$ processes respectively model the output and input
of a termination signal on the channel $x$. The former process has no
continuation, while the latter process continues as $P$ once the signa has been
received.
%
The $\Fail\x$ process denotes a runtime error concerning the channel $x$.
%
The $\Select\x{\InTag_i}\z.P$ and $\Case\x\z{Q_1}{Q_2}$ processes respectively
denote the output and input of a label $\InTag_i$ along with a fresh
continuation $z$ on the channel $x$. Depending on the label $\InTag_i$, the
receiver continues as $Q_i$. In practice it is useful to consider arbitrary
(domain-specific) sets of tags instead of the canonical two $\InTag_1$ and
$\InTag_2$. This generalization has no substantial impact on the formalization,
so we stick to the classical case of ``binary choices'' in line with the usual
presentations of linear logic.
%
The $\Fork\x\y\z{P}{Q}$ and $\Join\x\y\z.R$ processes respectively denote the
output and input of a pair of fresh channels $y$ and $z$ on the channel $x$. The
former process forks into $P$ and $Q$, each using $y$ and $z$ respectively. The
latter process continues as $R$. These forms are often used in conjunction with
links to model \emph{free outputs}, but can also be used to model bifurcating
sessions whereby a sequential protocol continues as two parallel sub-protocols.

Next we have process forms dealing with shared (non-linear) channels. The
$\Server\x\y.P$ and $\Client\x\y.P$ processes respectively denote \emph{servers}
and \emph{clients} acting on the $x$. Each request (from a client) spawns a copy
of the server's body using the continuation channel $y$. The process
$\Weaken\x.P$ denotes an explicit \emph{weakening}, that is a client that does
\emph{not} use the server waiting on $x$. The process $\Contract\x\y\z.P$
denotes an explicit \emph{contraction}, that is a client that uses the server
$x$ multiple times on the channels $y$ and $z$.

\emph{Cuts} of the form $\Cut\x{P}{Q}$ represent the parallel composition of the
processes $P$ and $Q$ connected by a channel $x$.

The notions of free and bound names are almost standard. Just bear in mind that
also output action prefixes bind continuation channels in (some) continuation
processes. In particular, $\Select\x{\Tag_i}\z.P$ binds $z$ in $P$, while
$\Fork\x\y\z{P}{Q}$ binds $y$ in $P$ but not in $Q$ and $z$ in $Q$ but not in
$P$.
%
We write $\fn{P}$ for the set of channel names occurring free in $P$ and we
identify processes up to renaming of bound names.

\subsection{Operational Semantics}
\label{sec:semantics}

\begin{table}
    \caption{Operational semantics of \Calculus.}
    \label{tab:semantics}
    \begin{tabular}{@{}c@{}}
        \begin{math}
            \displaystyle
            \begin{array}{rrcll}
            \SLink &
            \Link\x\y & \pcong & \Link\y\x \\
            \SComm &
            \Cut\x{P}{Q} & \pcong & \Cut\x{Q}{P} \\
            \SFail &
            \Cut\x{\Fail\y}{P} & \pcong & \Fail\y & x \ne y \\
            \SWait &
            \Cut\x{\Wait\y.P}{Q} & \pcong & \Wait\y.\Cut\x{P}{Q} & x \ne y \\
            \SSelect &
            \Cut\x{\Select\y{\InTag_i}\z.P}{Q} & \pcong & \Select\y{\InTag_i}\z.\Cut\x{P}{Q} & x \ne y,z \\
            \SCase &
            \Cut\x{\Case\y\z{P}{Q}}{R} & \pcong & \Case\y\z{\Cut\x{P}{R}}{\Cut\x{Q}{R}} & x \ne y,z \\
            \SForkL &
            \Cut\x{\Fork\y\u\v{P}{Q}}{R} & \pcong & \Fork\y\u\v{\Cut\x{P}{R}}{Q} & x \in \fn{P}\setminus\set{y,u,v} \\
            \SForkR &
            \Cut\x{\Fork\y\u\v{P}{Q}}{R} & \pcong & \Fork\y\u\v{P}{\Cut\x{Q}{R}} & x \in \fn{Q}\setminus\set{y,u,v} \\
            \SJoin &
            \Cut\x{\Join\y\u\v.P}{Q} & \pcong & \Join\y\u\v.\Cut\x{P}{Q} & x \ne y,u,v \\
            \SClient &
            \Cut\x{\Client\y\z.P}{Q} & \pcong & \Client\y\z.\Cut\x{P}{Q} & x \ne y,z \\
            \SWeaken &
            \Cut\x{\Weaken\y.P}{Q} & \pcong & \Weaken\y.\Cut\x{P}{Q} & x \ne y \\
            \SContract &
            \Cut\x{\Contract\y\u\v.P}{Q} & \pcong & \Contract\y\u\v.\Cut\x{P}{Q} & x \ne y,u,v \\
            \SServer &
            \Cut\x{\Server\y\u.P}{\Server\x\v.Q} & \pcong & \Server\y\u.\Cut\x{P}{\Server\x\v.Q} & x \ne y,u \\
            \\
            \RLink &
            \Cut\x{\Link\x\y}{P} & \red & P\subst\y\x \\
            \RClose &
            \Cut\x{\Close\x}{\Wait\x.P} & \red & P \\
            \RSelect &
            \Cut\x{\Select\x{\Tag_i}\z.P}{\Case\x\z{Q_1}{Q_2}} & \red & \Cut\z{P}{Q_i} & i\in\set{1,2} \\
            \RFork &
            \Cut\x{\Fork\x\y\z{P}{Q}}{\Join\x\y\z.R} & \red & \Cut\y{P}{\Cut\z{Q}{R}} \\
            \RConnect &
            \Cut\x{\Server\x\y.P}{\Client\x\y.Q} & \red & \Cut\y{P}{Q} \\
            \RWeaken &
            \Cut\x{\Server\x\y.P}{\Weaken\x.Q} & \red & \Weaken{\seqof\z}.Q
            & \fn{P} = \set{y,\seqof\z} \\
            \RContract &
            \Cut\x{\Server\x\y.P}{\Contract\x{x'}{x''}.Q} & \red &
            \multicolumn{2}{l}{
                \Contract{\seqof\z}{\seqof\z'}{\seqof\z''}.\Cut{x'}{
                    \Server{x'}\y.P'
                }{
                    \Cut{x''}{
                        \Server{x''}\y.P''
                    }{
                        Q
                    }
                }
                ~^*
            }
            \\
            \RCut &
            \Cut\x{P}{R} & \red & \Cut\x{Q}{R} & P \red Q \\
            \RCong &
            P & \red & Q & P \pcong R \red Q \\
            \\
            \multicolumn{5}{c}{
                ~^*
                \fn{P} = \set{y,\seqof\z},
                P' = P\subst{\seqof\z'}{\seqof\z},
                P'' = P\subst{\seqof\z''}{\seqof\z}
            }
            \end{array}
        \end{math}
    \end{tabular}
\end{table}

The operational semantics of \Calculus is shown in~\Cref{tab:semantics} and
is given by two relations: a \emph{structural pre-congruence} relation $\pcong$
and a \emph{reduction} relation $\red$. Structural pre-congruence relates
processes that we consider to be essentially indistinguishable. Reduction
relates processes in which a communication occurs. In some presentations of
linear logic~\cite{BaeldeDoumaneSaurin16,Doumane17}, these relations are
respectively dubbed \emph{external} and \emph{internal} reductions. Let us
describe the two relations more in detail.

Structural pre-congruence is the least pre-congruence defined by the
\rulename{s-*} rules.
%
\SLink and \SComm assert that links and parallel compositions are symmetric. The
remaining rules, when read from left to right, push a cut on $x$ underneath the
topmost prefix on $y$ of one of its sub-processes when $x \ne y$. These rules
are key to float input/output actions to the top-level of a process, so that
they can interact with corresponding complementary actions in the context
surrounding the process.
%
Note that some rules have additional side-conditions aimed at preserving the
meaning of names when binders are moved around. Also, there are two versions of
\SForkL and \SForkR depending on which of the two continuations $P$ and $Q$
contains occurrences of the restricted channel $x$.

Notably and somewhat surprisingly, there is no pre-congruence rule asserting
that parallel composition is associative. While this is a sound and expected
property of parallel composition, it turns out to be inessential for proving all
of the properties of \Calculus that we consider. We leave it out in light of the
forthcoming formalization of the calculus, in which every simplification is
welcome, although standard presentations of classical linear logic admit such
reduction~\cite{Doumane17}.
%
Another rule that deserves attention is \SServer. In this case, a server prefix
$\Server\y\u$ can be pulled out of a cut only provided that the other process in
the cut is also a server on the channel restricted by the cut.

The reduction rules \RLink, \RClose, \RSelect, \RFork, \RConnect, \RWeaken and
\RContract are in one-to-one correspondence with the well-known reductions of
session calculi and the principal cut reductions of linear logic. \RLink
eliminates a link $\Link\x\y$ by substituting $y$ for $x$. \RClose closes a
session $x$ when a termination signal is exchanged on $x$. \RSelect and \RFork
respectively model the communication of a tag $\Tag_i$ and of a fresh channel
$y$. Each of these communications carries along a \emph{fresh continuation} $z$
on which the rest of the conversation takes place, in line with the semantics of
the linear $\pi$-calculus~\cite{KobayashiPierceTurner99}.
%
\RConnect, \RWeaken and \RContract describe interactions on shared channels and
respectively represent the connection between a server and a client, the
disposal of an unused server, and the duplication of a server. In \RWeaken and
\RContract we use some convenient (albeit slightly ambiguous) notation for
denoting sequences of (pairwise distinct) channels and prefixes. In particular,
$\seqof\z$ stands for a sequence $z_1, \dots, z_n$ of channels,
$\Weaken{\seqof\z}.Q$ stands for $\Weaken{z_1}\dots\Weaken{z_n}.Q$ and
$\Contract{\seqof\z}{\seqof\z'}{\seqof\z''}.R$ stands for
$\Contract{z_1}{z_1'}{z_1''}\dots\Contract{z_n}{z_n'}{z_n''}.R$.

The rule~\RCut propagates reductions through cuts and the rules \RCong closes
reduction by structural pre-congruence.

\subsection{Type System}
\label{sec:typing-rules}

The types for \Calculus, ranged over by $A$, $B$, \dots, are linear logic
 propositions generated by the following grammar:
\[
    A, B ::= \Zero \mid \Top \mid \One \mid \Bot \mid A \Plus B \mid A \With B \mid A \Ten B \mid A \Par B \mid \OfCourse A \mid \WhyNot A
\]

The interpretation of propositions as session types is the usual
one~\cite{Wadler14}: the constants $\Bot$ and $\One$ describe the behavior of a
process sending/receiving a termination signal; the additive connectives $A
\Plus B$ and $\A \With B$ describe the behavior of a process sending/receiving a
boolean value and then adhering to either $A$ and $B$ accordingly; the
multiplicative connectives $A \Ten B$ and $A \Par B$ describe the behavior of a
process sending/receiving a channel (to be used according to $A$) and then
continuing as described by $B$. The additive constants $\Zero$ and $\Top$ do not
describe useful protocols, but they can serve the role of smallest/largest type
in type systems that support a notion of subtyping~\citep{HornePadovani24}. The
``of course'' modality $\OfCourse A$ and the ``why not'' modality $\WhyNot A$
describe the behavior of servers and clients accepting and requesting
connections to be used according to $A$.

The notion of \emph{duality} is standard and we write $\dual{A}$ for the dual of
$A$.

Typing contexts (\ie sequents) are finite maps from channel names to types
written as $x_1 : A_1, \dots, x_n : A_n$ and ranged over by $\ContextC$ and
$\ContextD$. We write $\ContextC, \ContextD$ for the union of $\ContextC$ and
$\ContextD$ when they have disjoint domain. We write $\WhyNot\Context$ for some
context $\Context = x_1 : \WhyNot A_1, \dots, x_n : \WhyNot A_n$ where all the
types in its range are prefixed by the $\WhyNot$ modality.

\begin{table}
    \caption{Typing rules of \Calculus.}
    \label{tab:typing-rules}
    \begin{tabular}{@{}c@{}}
        \begin{mathpar}
            \inferrule[\LinkRule]{~}{
                \wtp{\Link\x\y}{x : A, y : \dual{A}}
            }
            \and
            \inferrule[\FailRule]{~}{
                \wtp{\Fail\x}{x : \Top, \Context}
            }
            \and
            \inferrule[\CloseRule]{~}{
                \wtp{\Close\x}{x : \One}
            }
            \and
            \inferrule[\WaitRule]{
                \wtp{P}{\Context}
            }{
                \wtp{\Wait\x.P}{x : \Bot, \Context}
            }
            \and
            \inferrule[\SelectRule]{
                \wtp{P}{y : A_i, \Context}
            }{
                \wtp{\Select\x{\InTag_i}\y.P}{x : A_1 \Plus A_2, \Context}
            }
            \and
            \inferrule[\CaseRule]{
                \wtp{P}{y : A, \Context}
                \\
                \wtp{Q}{y : B, \Context}
            }{
                \wtp{\Case\x\y{P}{Q}}{x : A \With B, \Context}
            }
            \and
            \inferrule[\ForkRule]{
                \wtp{P}{y : A, \ContextC}
                \\
                \wtp{Q}{z : B, \ContextD}
            }{
                \wtp{\Fork\x\y\z{P}{Q}}{x : A \Ten B, \ContextC, \ContextD}
            }
            \and
            \inferrule[\JoinRule]{
                \wtp{P}{y : A, z : B, \Context}
            }{
                \wtp{\Join\x\y\z.P}{x : A \Par B, \Context}
            }
            \and
            \inferrule[\ServerRule]{
                \wtp{P}{y : A, \WhyNot\Context}
            }{
                \wtp{\Server\x\y.P}{x : \OfCourse A, \WhyNot\Context}
            }
            \and
            \inferrule[\ClientRule]{
                \wtp{P}{y : A, \Context}
            }{
                \wtp{\Client\x\y.P}{x : \WhyNot A, \Context}
            }
            \and
            \inferrule[\WeakenRule]{
                \wtp{P}\Context
            }{
                \wtp{P}{x : \WhyNot A, \Context}
            }
            \and
            \inferrule[\ContractRule]{
                \wtp{P}{y : \WhyNot A, z : \WhyNot A, \Context}
            }{
                \wtp{P}{x : \WhyNot A, \Context}
            }
            \and
            \inferrule[\CutRule]{
                \wtp{P}{x : A, \ContextC}
                \\
                \wtp{Q}{x : \dual{A}, \ContextD}
            }{
                \wtp{\Cut\x{P}{Q}}{\ContextC, \ContextD}
            }
        \end{mathpar}
    \end{tabular}
\end{table}

Typing judgments have the form $\wtp{P}\Context$ meaning that the process $P$ is
well typed in the context $\Context$. Equivalently, the judgment indicates that
the sequent $\wtp{}\Context$ is derivable and $P$ is a \emph{proof term}
corresponding to the derivation for $\wtp{}\Context$.
%
The typing rules are shown in \Cref{tab:typing-rules}. Since they are in
one-to-one correspondence with the proof rules of classical linear logic, the
reader may refer to the standard literature on linear logic~\cite{} or
propositions as sessions~\citep{Wadler14} for their interpretation.

\subsection{Properties of Well-Typed Processes}
\label{sec:properties}

The \emph{linearity challenge}~\citep{CarboneEtAl24} aims at formalizing two
essential properties of well-typed processes: (1) typing is preserved by
reductions; (2) peer endpoints are always used in complementary ways. This
latter property is called \emph{well formedness} in the challenge. In this work
we also consider \emph{deadlock freedom}, which is more general than well
formedness and holds for \Calculus since its type system is based on linear
logic. We now formulate these properties using the notation developed so far.

Concerning the preservation of typing, this corresponds to the usual subject
reduction result, which is expressed thus:

\begin{theorem}[subject reduction]
    \label{thm:red}
    If $P \red Q$ then $\wtp{P}\Context$ implies $\wtp{Q}\Context$.
\end{theorem}

In order to formulate deadlock freedom, we first need to introduce some
terminology for referring to processes that are unable to make any progress. A
simple example of deadlock is the process $\Cut\x{\Close\x}{\Close\x}$. This
process is unable to reduce (because a $\Close\x$ process is meant to interact
with a process of the form $\Wait\x.P$) and, more generally, it is unable to
interact regardless of the context in which it occurs because the $\Close\x$
sub-processes it contains are blocked on the channel $x$ bound by the cut.
%
In general, the property of being a deadlock is not simply the inability to
reduce: there are irreducible processes that are not a deadlock because they
would be able to reduce if put in a suitable context. For example, $\Wait\x.P$
does not reduce, and yet it is not a deadlock because it would be able to make
progress when composed in parallel with $\Close\x$. In general, even a process
like $\Cut\y{\Wait\y.P}{\Wait\z.Q}$ where $x\ne y,z$ cannot be considered a
deadlock, because the prefixes $\Wait\y$ and $\Wait\z$ could be exposed using
\SWait and possibly \SComm. Let us make all this more precise.

We say that $P$ is a \emph{thread} if it is anything but a cut. In other words,
a thread is either a link or a process that starts with an input/output action
of some sort. Note that a thread may contain cuts, but these cuts must all
guarded underneath the topmost action prefix.
%
We also say that $P$ is \emph{observable} if $P \pcong Q$ for some thread $Q$.
An observable process is a process that exposes an action on a free channel name
and therefore can interact through that channel, if put into some appropriate
context. We say that $P$ is \emph{reducible} if $P \red Q$ for some $Q$. A
reducible process may perform a reduction step.
%
We say that a process is \emph{alive} if it is either observable or reducible
and that it is a \emph{deadlock} if it is not alive.

\begin{theorem}[deadlock freedom]
    \label{thm:df}
    If $\wtp{P}\Context$ then $P$ is alive.
\end{theorem}

In order to formulate \emph{well formedness}~\cite{CarboneEtAl24}, we define
\emph{reduction contexts} as partial processes with a single unguarded hole
$\Hole$, thus:
\[
    \ReductionContext ::= \Hole \mid \Cut\x\ReductionContext{P} \mid \Cut\x{P}\ReductionContext
\]

As usual, we write $\ReductionContext[P]$ for the process obtained by replacing
the hole in $\ReductionContext$ with $P$, noting that such replacement may
capture channel names that are bound in $\ReductionContext$ and occur free in
$P$.
%
Now we observe that if $P$ and $Q$ both act on the same channel $x$ in
complementary ways, then their parallel composition $\Cut\x{P}{Q}$ is reducible
according to one of the principal cut reductions described in
\Cref{tab:semantics}. Therefore, an alternative (and more general) way of
formulating well formedness is simply this:
%
we say that $P$ is \emph{well formed} if $P \pcong \ReductionContext[Q]$ implies
that $Q$ is alive.

\begin{theorem}[type safety]
    \label{thm:type-safety}
    If $\wtp{P}\Context$ then $P$ is well formed.
\end{theorem}

Note that the properties expressed in \Cref{thm:df,thm:type-safety} are
invariant under reductions thanks to \Cref{thm:red}.


\bibliography{main}

\end{document}